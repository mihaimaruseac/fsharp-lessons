\documentclass{beamer}

\usepackage{hyperref}
\mode<presentation>

\title{F\#.2 : Lists and Sequences. Higher Order Functions}
\author{Mihai Maruseac, ROSEdu\\mihai@rosedu.org}

\setbeamertemplate{frametitle continuation}[from second]
\setbeamertemplate{footline}[frame number]

\begin{document}

\maketitle

\begin{frame}
  \tableofcontents
\end{frame}

\begin{frame}[fragile]{Last time...}
  \begin{itemize}
    \item short history of FP
    \item F\#: functional, imperative, object-oriented
    \item types in F\#
    \item functions and pattern matching
    \item recursive functions
    \item lists and tuples
  \end{itemize}
\end{frame}

\begin{frame}[fragile]{Last Assignment}
  \begin{itemize}
    \item Implement a \textit{Caesar cipher}
    \item For input \texttt{cipher "hello" 13}
    \item Should produce \texttt{uryyb}
  \end{itemize}
\end{frame}

\begin{frame}[fragile]{Solution (1)}
        \tiny
  \begin{verbatim}
> let letter_to_int (c:char) = (int c) - (int 'a');;
> let int_to_letter v = char ((v % 26 + (int 'a')));;
> let rec convert_to_string s = match s with
                                | [] -> ""
                                | l::ls -> l.ToString() + convert_to_string ls;;
> let convert_to_list str = List.ofSeq str;;
> let convert_letter letter delta = int_to_letter ((letter_to_int letter) + delta);;
> let rec convert_each letters delta = match letters with
                                       | [] -> []
                                       | l::ls -> convert_letter l delta :: convert_each ls delta;;
> let cipher input delta = convert_to_string (convert_each (convert_to_list input) delta);;
> cipher "hello" 13;;
val it : string = "uryyb"
  \end{verbatim}
\end{frame}

\begin{frame}[fragile]{Solution(2)}
        \tiny
  \begin{verbatim}
> let cipher (input:string) delta = 
      input |> 
      List.ofSeq |>
      List.map (fun x -> char ((delta + int x - int 'a') % 26 + int 'a')) |>
      List.fold (fun x y -> x + y.ToString()) "";;
  \end{verbatim}
\end{frame}

\section{Lists and Sequences}
\frame{\tableofcontents[currentsection]}

\begin{frame}[fragile]{List Construction}
  \begin{itemize}
    \item element-wise: \texttt{2 :: []}
    \item ranges (list comprehensions): \texttt{[0 .. 5]}
    \item ranges with steps: \texttt{[0 .. 2 .. 10]}
    \item initializer: \texttt{List.init 5 square}
    \item generators
    \begin{verbatim}
    > [ for a in 1 .. 10 do
            yield (a * a) ];;
    \end{verbatim}
    \pause
    \begin{verbatim}
    > [ for a in 1 .. 3 do
        for b in 3 .. 7 do
            yield (a, b) ];;
    \end{verbatim}
  \end{itemize}
\end{frame}

\begin{frame}[fragile]{Generators}
  \begin{verbatim}
  > [ for a in 1 .. 100 do
      if a % 3 = 0 && a % 5 = 0 then yield a];;
  \end{verbatim}
  \pause
  \begin{verbatim}
  > [for a in ["ana", "maria"] do yield [a; a; a] ];;
  \end{verbatim}
  \pause
  \begin{verbatim}
  > [for a in ["ana", "maria"] do yield! [a; a; a] ];;
  \end{verbatim}
\end{frame}

\begin{frame}[fragile]{Mixing \texttt{yield} and \texttt{yield!}}
  \begin{verbatim}
  > [for a in 1 .. 5 do
      match a with
      | 3 -> yield! ["hello"; "world"]
      | _ -> yield a.ToString() ];;
  \end{verbatim}
\end{frame}

\begin{frame}[fragile]{Infinite Data Types}
  \begin{itemize}[<+->]
    \item infinite streams of values
    \item finite representation
    \item F\# is eager. Haskell is lazy
    \item sequences
  \end{itemize}
  \pause
  \tiny
  \begin{verbatim}
  > seq { 1 .. 10 };;
  > seq { 1 .. 2 .. 10 };;
  > seq { 10 .. -1 .. 0 };;
  > seq { for a in 1 .. 10 do yield a, a*a, a*a*a };;
  > let allEvens =
      let rec loop x = seq { yield x; yield! loop (x + 2) }
      loop 0;;
  > let seqInfinite = Seq.initInfinite (fun index ->
        let n = float( index + 1 )
        1.0 / (n * n * (if ((index + 1) % 2 = 0) then 1.0 else -1.0)));;
  \end{verbatim}
\end{frame}

\begin{frame}{Bread and Butter Modules}
  \begin{description}
    \item[lists] \texttt{List.}
    \item[sequences] \texttt{Seq.}
  \end{description}
  \pause
  \begin{itemize}
    \item create new structures
    \item extract elements
    \item skip elements
    \item code reduction
    \item code readability
    \item remember DRY principle
    \item proving correctness
  \end{itemize}
\end{frame}

\begin{frame}[fragile]{Take}
  \begin{itemize}
    \item \texttt{Seq.take : int -> seq<'a> -> seq<'a>}
  \end{itemize}
  \tiny
  \begin{verbatim}
  > Seq.take 5 (Seq.initInfinite (fun x -> x * x));;
  val it : seq<int> = seq [0; 1; 4; 9; ...]
  > Seq.toList (Seq.take 5 (Seq.initInfinite (fun x -> x * x)));;
  val it : int list = [0; 1; 4; 9; 16]
  > (fun x -> x * x) |> Seq.initInfinite |> Seq.take 5 |> Seq.toList;;
  val it : int list = [0; 1; 4; 9; 16]
  \end{verbatim}
\end{frame}

\begin{frame}[fragile]{Drop}
  \begin{itemize}
    \item \texttt{Seq.skip : int -> seq<'a> -> seq<'a>}
  \end{itemize}
  \begin{verbatim}
  > Seq.skip 5 (Seq.initInfinite (fun x -> x * x));;
  val it : seq<int> = seq [25; 36; 49; 64; ...]
  \end{verbatim}
\end{frame}

\begin{frame}{FizzBuzz}
  \begin{block}{Exercise}
  Write a program that prints the numbers from 1 to 100. But for multiples of
  three print "Fizz" instead of the number and for the multiples of five print
  "Buzz". For numbers which are multiples of both three and five print
  "FizzBuzz".
  \end{block}
\end{frame}

\begin{frame}[fragile]{Solution with Lists}
  \tiny
  \begin{verbatim}
> [ for i in 1 .. 100 do
      if i % 15 = 0 then yield "FizzBuzz"
      elif i % 5 = 0 then yield "Buzz"
      elif i % 3 = 0 then yield "Fizz"
      else yield i.ToString() ];;
val it : string list =
  ["1"; "2"; "Fizz"; "4"; "Buzz"; "Fizz"; "7"; "8"; "Fizz"; "Buzz"; "11";
   "Fizz"; "13"; "14"; "FizzBuzz"; "16"; "17"; "Fizz"; "19"; "Buzz"; "Fizz";
   "22"; "23"; "Fizz"; "Buzz"; "26"; "Fizz"; "28"; "29"; "FizzBuzz"; "31";
   "32"; "Fizz"; "34"; "Buzz"; "Fizz"; "37"; "38"; "Fizz"; "Buzz"; "41";
   "Fizz"; "43"; "44"; "FizzBuzz"; "46"; "47"; "Fizz"; "49"; "Buzz"; "Fizz";
   "52"; "53"; "Fizz"; "Buzz"; "56"; "Fizz"; "58"; "59"; "FizzBuzz"; "61";
   "62"; "Fizz"; "64"; "Buzz"; "Fizz"; "67"; "68"; "Fizz"; "Buzz"; "71";
   "Fizz"; "73"; "74"; "FizzBuzz"; "76"; "77"; "Fizz"; "79"; "Buzz"; "Fizz";
   "82"; "83"; "Fizz"; "Buzz"; "86"; "Fizz"; "88"; "89"; "FizzBuzz"; "91";
   "92"; "Fizz"; "94"; "Buzz"; "Fizz"; "97"; "98"; "Fizz"; "Buzz"]
  \end{verbatim}
\end{frame}

\begin{frame}[fragile]{Solution with Sequences (1)}
  \begin{verbatim}
> seq { for i in 1 .. 100 do
      if i % 15 = 0 then yield "FizzBuzz"
      elif i % 5 = 0 then yield "Buzz"
      elif i % 3 = 0 then yield "Fizz"
      else yield i.ToString() };;
val it : seq<string> = seq ["1"; "2"; "Fizz"; "4"; ...]
  \end{verbatim}
\end{frame}

\begin{frame}[fragile]{Solution with Sequences (2)}
  \begin{verbatim}
> Seq.take 100 (Seq.skip 1 (Seq.initInfinite (fun x -> 
          if x % 15 = 0 then "FizzBuzz"
          elif x % 5 = 0 then "Buzz"
          elif x % 3 = 0 then "Fizz"
          else x.ToString())));;
val it : seq<string> = seq ["1"; "2"; "Fizz"; "4"; ...]
  \end{verbatim}
\end{frame}

\begin{frame}[fragile]{Appending}
  \tiny
  \begin{itemize}
    \item \texttt{List.append : val it : 'a list -> 'a list -> 'a list}
    \item \texttt{(@) : val it : 'a list -> 'a list -> 'a list}
    \item \texttt{Seq.append : val it : seq<'a> -> seq<'a> -> seq<'a>}
  \end{itemize}
  \begin{verbatim}
  > List.append [1;2;3] [4;5;6];;
  val it : int list = [1; 2; 3; 4; 5; 6]
  > [1;2;3] @ [4;5;6];;
  val it : int list = [1; 2; 3; 4; 5; 6]
  \end{verbatim}
\end{frame}

\begin{frame}[fragile]{Extracting elements}
  \begin{itemize}
    \item \texttt{List.head : 'a list -> 'a}
    \item \texttt{Seq.head : seq<'a> -> 'a}
    \item \texttt{List.nth : 'a list -> int -> 'a}
    \item \texttt{Seq.nth : int -> seq<'a> -> 'a}
  \end{itemize}
  \begin{verbatim}
  > Seq.head (Seq.initInfinite (fun x -> x * x));;
  val it : int = 0
  > Seq.nth 15 (Seq.initInfinite (fun x -> x * x));;
  val it : int = 225
  \end{verbatim}
\end{frame}

\begin{frame}[fragile]{Tails}
  \begin{itemize}
    \item \texttt{List.tail : ('a list -> 'a list)}
    \item \textbf{Only for lists!}
  \end{itemize}
  \begin{verbatim}
  > List.tail [1;2;3;4];;
  val it : int list = [2; 3; 4]
  \end{verbatim}
\end{frame}

\begin{frame}[fragile]{Windows}
  \begin{itemize}
    \item \texttt{Seq.windowed : (int -> seq<'a> -> seq<'a []>)}
    \item \textbf{Only for sequences!}
    \item Returns sequence of arrays
  \end{itemize}
  \tiny
  \begin{verbatim}
  > Seq.windowed 2 (Seq.initInfinite (fun x -> x));;
  val it : seq<int []> = seq [[|0; 1|]; [|1; 2|]; [|2; 3|]; [|3; 4|]; ...]
  > Seq.windowed 2 (Seq.initInfinite (fun x -> x)) |> Seq.head;;
  val it : int [] = [|0; 1|]
  \end{verbatim}
\end{frame}

\begin{frame}[fragile]{Pairing and Unpairing}
  \small
  \begin{itemize}
    \item \texttt{List.zip : ('a list -> 'b list -> ('a * 'b) list)}
    \item \texttt{Seq.zip : (seq<'a> -> seq<'b> -> seq<'a * 'b>)}
    \item \texttt{List.unzip : (('a * 'b) list -> 'a list * 'b list)}
  \end{itemize}
  \tiny
  \begin{verbatim}
  > List.zip [2;3;4] ["ana"; "has"; "apples"];;
  val it : (int * string) list = [(2, "ana"); (3, "has"); (4, "apples")]
  > List.zip [2;3;4] ["ana"; "has"; "apples"] |> List.unzip;;
  val it : int list * string list = ([2; 3; 4], ["ana"; "has"; "apples"])
  \end{verbatim}
\end{frame}

\section{Higher Order Functions}
\frame{\tableofcontents[currentsection]}

\begin{frame}[fragile]{Higher Order Functions}
  \begin{itemize}[<+->]
    \item functions as data (parameters, returned)
    \item math: $lim$, $\nabla$, $\frac{d}{dx}$, $\frac{\delta}{\delta x}$
  \end{itemize}
  \pause
  \begin{verbatim}
  > let passAnswer f = f 42;;
  > let square x = x * x;;
  > let cube x = x * x * x;;
  > passAnswer square;;
  > passAnswer cube;;
  \end{verbatim}
\end{frame}

\begin{frame}[fragile]{Curry vs Uncurry Functions}
  \begin{verbatim}
  int f(int x, int y) { ... }
  ....
    f(3);
  ....
  \end{verbatim}
  \pause
  \begin{verbatim}
  > let f x y = x + y;;
  > f 3;;
  \end{verbatim}
  \pause
  \begin{verbatim}
  > let addAnswer = f 42;;
  \end{verbatim}
\end{frame}

\begin{frame}{Function Sequencing}
  \begin{itemize}
    \item \texttt{|>}
    \item \texttt{>>}
    \item \texttt{<<}
  \end{itemize}
\end{frame}

\begin{frame}[fragile]{Feed-Forward}
  \begin{verbatim}
  let inline (|>) x f = f x
  \end{verbatim}
  \begin{verbatim}
  > 42 |>
    (fun x -> x + 1) |>
    (fun x -> x * x) |>
    (fun x -> x - 1800) |>
    (fun x -> x.ToString());;
  \end{verbatim}
\end{frame}

\begin{frame}[fragile]{Compositions}
  \begin{verbatim}
  let inline (>>) f g x = g (f x)
  let inline (<<) f g x = f (g x)
  \end{verbatim}
  \begin{verbatim}
  > let f x = printf "f called with %A\n" x; x - 1;;
  > let g x = printf "g called with %A\n" x; x + 1;;
  > (f >> g) 4;;
  f called with 4
  g called with 3
  val it : int = 4
  > (f << g) 4;;
  g called with 4
  f called with 5
  val it : int = 4
  \end{verbatim}
\end{frame}

\section{Higher Order Functions for Lists and Sequences}
\frame{\tableofcontents[currentsection]}

\begin{frame}[fragile]{Mapping}
  \tiny
  \begin{itemize}
    \item \texttt{List.map : (('a -> 'b) -> 'a list -> 'b list)}
    \item \texttt{Seq.map : ((int -> 'a -> 'b) -> 'a list -> 'b list)}
    \item \texttt{List.mapi : (('a -> 'b) -> seq<'a> -> seq<'b>)}
    \item \texttt{Seq.mapi : ((int -> 'a -> 'b) -> seq<'a> -> seq<'b>)}
  \end{itemize}
  \begin{verbatim}
  > List.map (fun x -> x.ToString()) [1 .. 5];;
  val it : string list = ["1"; "2"; "3"; "4"; "5"]
  \end{verbatim}
\end{frame}

\begin{frame}[fragile]{Filtering}
  \tiny
  \begin{itemize}
    \item \texttt{List.filter : (('a -> bool) -> 'a list -> 'a list)}
    \item \texttt{Seq.filter : (('a -> bool) -> seq<'a> -> seq<'a>)}
  \end{itemize}
  \begin{verbatim}
  > List.filter (fun x -> x % 3 = 0) [1 .. 5];;
  val it : int list = [3]
  \end{verbatim}
\end{frame}

\begin{frame}[fragile]{Folding}
  \tiny
  \begin{itemize}
    \item \texttt{List.fold : (('a -> 'b -> 'a) -> 'a -> 'b list -> 'a)}
    \item \texttt{Seq.fold : (('a -> 'b -> 'a) -> 'a -> seq<'b> -> 'a)}
    \item \texttt{List.foldBack: (('a -> 'b -> 'b) -> 'a list -> 'b -> 'b)}
  \end{itemize}
  \begin{verbatim}
  > List.fold (*) 1 [1 .. 5];;
  val it : int = 120
  > List.fold (-) 100 [1 .. 5];;
  val it : int = 85
  > ((((100 - 1) - 2) - 3) - 4) - 5;;
  val it : int = 85
  > List.foldBack (-) [1 .. 5] 100;;
  val it : int = -97
  > 1 - (2 - (3 - (4 - (5 - 100))));;
  val it : int = -97
  \end{verbatim}
\end{frame}

\begin{frame}{Reductions}
  \small
  \begin{itemize}
    \item \texttt{List.reduce : (('a -> 'a -> 'a) -> 'a list -> 'a)}
    \item \texttt{Seq.reduce : (('a -> 'a -> 'a) -> seq<'a> -> 'a)}
    \item \texttt{List.reduceBack : (('a -> 'a -> 'a) -> 'a list -> 'a)}
  \end{itemize}
  \begin{itemize}
    \item folds starting with the first/last element
  \end{itemize}
\end{frame}

\begin{frame}[fragile]{Scans}
  \tiny
  \begin{itemize}
    \item \texttt{List.scan : (('a -> 'b -> 'a) -> 'a -> 'b list -> 'a list)}
    \item \texttt{List.scanBack : (('a -> 'b -> 'b) -> 'a list -> 'b -> 'b list)}
    \item \texttt{Seq.scan : (('a -> 'b -> 'a) -> 'a -> seq<'b> -> seq<'a>)}
  \end{itemize}
  \begin{verbatim}
  > List.scan (-) 100 [1 .. 5];;
  val it : int list = [100; 99; 97; 94; 90; 85]
  > List.scanBack (-) [1 .. 5] 100;;
  val it : int list = [-97; 98; -96; 99; -95; 100]
  \end{verbatim}
\end{frame}

\begin{frame}[fragile]{Collecting}
  \begin{itemize}
    \item \texttt{map} returns a list
    \item want to concatenate the results
    \pause
    \tiny
    \item \texttt{List.collect : (('a -> 'b list) -> 'a list -> 'b list}
    \item \texttt{Seq.collect : (('a -> \#seq<'c>) -> seq<'a> -> seq<'c>)}
  \end{itemize}
  \begin{verbatim}
  > List.collect (fun x -> [x; x+1]) [1; 2; 3];;
  val it : int list = [1; 2; 2; 3; 3; 4]
  > List.map (fun x -> [x; x+1]) [1; 2; 3];;
  val it : int list list = [[1; 2]; [2; 3]; [3; 4]]
  \end{verbatim}
\end{frame}

\begin{frame}[fragile]{Predicate Satisfaction}
  \begin{itemize}
    \item \texttt{List.exists : (('a -> bool) -> 'a list -> bool)}
    \item \texttt{Seq.exists : (('a -> bool) -> seq<'a> -> bool)}
    \item \texttt{List.forall : (('a -> bool) -> 'a list -> bool)}
    \item \texttt{Seq.forall : (('a -> bool) -> seq<'a> -> bool)}
  \end{itemize}
  \begin{verbatim}
  > List.exists (fun x -> x % 2 = 0) [1; 2; 3];;
  val it : bool = true
  > List.forall (fun x -> x < 5) [1; 2; 3];;
  val it : bool = true
  \end{verbatim}
\end{frame}

\begin{frame}[fragile]{Finding}
  \begin{itemize}
    \item \texttt{List.find : (('a -> bool) -> 'a list -> 'a)}
    \item \texttt{Seq.find : (('a -> bool) -> seq<'a> -> 'a)}
  \end{itemize}
  \begin{verbatim}
  > List.find (fun x -> x % 2 = 0) [1; 2; 3];;
  val it : int = 2
  \end{verbatim}
\end{frame}

\begin{frame}[fragile]{Selections by using Predicates}
  \tiny
  \begin{itemize}
    \item \texttt{Seq.takeWhile : (('a -> bool) -> seq<'a> -> seq<'a>)}
    \item \texttt{Seq.skipWhile : (('a -> bool) -> seq<'a> -> seq<'a>)}
  \end{itemize}
  \begin{verbatim}
  > Seq.skipWhile (fun x -> x < 5) (Seq.initInfinite (fun x -> x));;
  val it : seq<int> = seq [5; 6; 7; 8; ...]
  > Seq.takeWhile (fun x -> x * x > x + 5) (Seq.initInfinite (fun x -> x));;
  val it : seq<int> = seq []
  > Seq.takeWhile (fun x -> x * x < x + 5) (Seq.initInfinite (fun x -> x));;
  val it : seq<int> = seq [0; 1; 2]
  \end{verbatim}
\end{frame}

\section{Pointfree style}
\frame{\tableofcontents[currentsection]}

\begin{frame}{Pointfree style}
  \begin{itemize}[<+->]
    \item skip the argument (\textit{point}) where function is evaluated
    \item \texttt{let isNotError v = not (isError v)}
    \item \texttt{let isNotError1 v = (not << isError) v}
    \item \texttt{let isNotError2 v = isError v |> not}
    \item \texttt{let isNotError4 = isError >> not}
    \item \texttt{let isNotError5 = not << isError}
    \item \texttt{isNotError4}, \texttt{isNotError5} emitted as CLR properties
    \item the others are emitted as CLR functions
    \item F\# has a slight implementation distinction between data and functions
  \end{itemize}
\end{frame}

\begin{frame}{Advantages of Point-Free style}
  \begin{itemize}[<+->]
    \item DRY
    \item faster (\textit{partial function application})
    \item assemble low-level functions from higher-order ones on the fly
    \item skipping boilerplate
    \item \textbf{strong} focus on \textit{what} not \textit{how}
  \end{itemize}
\end{frame}

\section{Finishing slides}
\frame{\tableofcontents[currentsection]}

\begin{frame}[fragile]{Back to Pattern Matching}
  \begin{itemize}
    \item $\exists$ shorter variant
  \end{itemize}
  \begin{verbatim}
  > let rec f = function
      | [] -> 0
      | _::xs -> 1 + f xs;;
  \end{verbatim}
  \pause
  Translated as
  \begin{verbatim}
  > let rec f = (fun x -> match x with
      | [] -> 0
      | _::xs -> 1 + f xs
      );;
  \end{verbatim}
\end{frame}

\begin{frame}{Recap}
  \begin{itemize}[<+->]
    \item functions are data
    \item F\# functions are curry
    \item lists and sequences
    \item common patterns captured in \texttt{List.} and \texttt{Seq.}
    functions
    \item point-free style
  \end{itemize}
\end{frame}

\begin{frame}[fragile]{Assignment}
  \begin{itemize}
    \item optional
    \item send via mail (\texttt{mihai@rosedu.org})
    \item next talk will start with discussion of solutions
  \end{itemize}
  \pause
  \begin{block}{Problem from Project Euler}
  A palindromic number reads the same both ways. The largest palindrome made
  from the product of two 2-digit numbers is 9009 = 91*99.

  Find the largest palindrome made from the product of two 3-digit numbers.
  \end{block}
\end{frame}

\end{document}
