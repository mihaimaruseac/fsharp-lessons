\documentclass{beamer}

\usepackage{hyperref}
\mode<presentation>

\title{F\#.3 : Immutable Data Structures}
\author{Mihai Maruseac, ROSEdu\\mihai@rosedu.org}

\setbeamertemplate{frametitle continuation}[from second]
\setbeamertemplate{footline}[frame number]

\begin{document}

\maketitle

\begin{frame}
  \tableofcontents
\end{frame}

\begin{frame}{Last time...}
  \begin{itemize}[<+->]
    \item list and sequences
    \item curry functions
    \item point-free programming
    \item \texttt{take}, \texttt{skip}
    \item \texttt{takeWhile}, \texttt{skipWhile}
    \item \texttt{map}
    \item \texttt{filter}
    \item \texttt{fold}
  \end{itemize}
\end{frame}

\begin{frame}{Last Assignment}
  \begin{block}{Problem from Project Euler}
  A palindromic number reads the same both ways. The largest palindrome made
  from the product of two 2-digit numbers is 9009 = 91*99.

  Find the largest palindrome made from the product of two 3-digit numbers.
  \end{block}
\end{frame}

\begin{frame}[fragile]{Solution (1)}
\small
  \begin{verbatim}
let rec digits = function
      | n when n < 10 -> [n]
      | n -> (n % 10) :: (digits (n / 10))
let revNumber = digits >> List.fold (fun x y -> y + 10 * x) 0
let isPalindrom x = x = revNumber x
[ for a in 999 .. -1 .. 100 do
    for b in 999 .. -1 .. 100 do
    let x = a * b
    if isPalindrom x then yield x,a,b] |> List.head;;
\end{verbatim}
\end{frame}

\section{Option Types}
\frame{\tableofcontents[currentsection]}

\begin{frame}{List searching}
  \begin{itemize}[<+->]
    \item \texttt{List.find : (('a -> bool) -> 'a list -> 'a)}
    \item What if element is not found in list?
    \item Runtime error...
    \item Fail if precondition is unmet
    \item Need way to signal that functions might fail
    \item \texttt{List.tryFind :\\ (('a -> bool) -> 'a list -> 'a option)}
  \end{itemize}
\end{frame}

\begin{frame}[fragile]{Option}
  \begin{itemize}
    \item \texttt{None} - no value to return
    \item \texttt{Some x} - result available
  \end{itemize}
  \pause
  \begin{verbatim}
  > let safediv x y =
          match y with
              | 0 -> None
              | _ -> Some(x/y);;
  \end{verbatim}
\end{frame}

\begin{frame}[fragile]{Option is not \texttt{null}}
  \begin{itemize}
    \item conceptually similar
    \item different semantics
    \item programmer \textbf{forced} to treat both cases
  \end{itemize}
  \begin{verbatim}
  val div : int -> int -> int
  val safediv : int -> int -> int option
  \end{verbatim}
\end{frame}

\begin{frame}[fragile,label=maybe]{Pattern Matching}
  \begin{verbatim}
  > let isFortyTwo = function
        | Some(42) -> true
        | Some(_) -> false
        | None -> false;;
  \end{verbatim}
\end{frame}

\begin{frame}{Other Functions}
  \begin{itemize}[<+->]
    \item \texttt{get : 'a option -> 'a}
    \item \texttt{isNone, isSome : 'a option -> bool}
    \item \texttt{map : ('a -> 'b) -> 'a option -> 'b option}
    \item \texttt{iter : ('a -> unit) -> 'a option -> unit}
  \end{itemize}
\end{frame}

\section{Records}
\frame{\tableofcontents[currentsection]}

\begin{frame}[label=tuples]{Tuples}
  \begin{itemize}
    \item \texttt{(3.14, "Ana", 42)}
    \item semantics?
    \item extracting values?
    \item cloning?
    \item type safety? (\texttt{point2D} vs \texttt{pair})
  \end{itemize}
\end{frame}

\begin{frame}[fragile]{Records}
  \begin{itemize}
    \item giving name to values in tuples
    \item method to properly extract values
  \end{itemize}
  \begin{verbatim}
  > type website = { Title : string; Url : string };;
  > let homepage = { Title = "Google";
        Url = "http://www.google.com" };;
  > { Url = "http://www.microsoft.com/";
        Title = "Microsoft Corporation" };;
  > website.Title;;
  \end{verbatim}
\end{frame}

\begin{frame}[fragile]{Cloning Records}
  \begin{itemize}
    \item immutable
    \item changing = creating a clone
  \end{itemize}
  \begin{verbatim}
  > type coords = { X : float; Y : float };;
  > let start = { X = 1.0; Y = 2.0 };;
  > let end = { start with X = 15.5 };;
  \end{verbatim}
\end{frame}

\begin{frame}[fragile]{Pattern Matching}
  \begin{verbatim}
type coords = { X : float; Y : float }
 
let getQuadrant = function
    | { X = 0.0; Y = 0.0 } -> "Origin"
    | item when item.X >= 0.0 && item.Y >= 0.0 -> "I"
    | item when item.X <= 0.0 && item.Y >= 0.0 -> "II"
    | item when item.X <= 0.0 && item.Y <= 0.0 -> "III"
    | item when item.X >= 0.0 && item.Y <= 0.0 -> "IV"
  \end{verbatim}
\end{frame}

\section{Sets and Maps}
\frame{\tableofcontents[currentsection]}

\begin{frame}{Sets}
  \begin{itemize}[<+->]
    \item container for items
    \item no duplicates
    \item no order guarantee
    \item \texttt{Set.empty.Add(1).Add(2).Add(7)}
    \item \texttt{Set.ofList ["Mercury"; "Venus"; "Earth"; "Mars";]}
    \item functions in \texttt{Set} module (\texttt{map}, \texttt{union},
    \texttt{contains}, $\ldots$)
  \end{itemize}
\end{frame}

\begin{frame}[fragile]{Maps}
  \begin{itemize}[<+->]
    \item special sets
    \item \textit{key} - \textit{value} pairs
  \end{itemize}
  \pause
  \begin{verbatim}
> let holidays =
    Map.empty.
        Add("Christmas", "Dec. 25").
        Add("Halloween", "Oct. 31");;
> let monkeys = 
    [ "Squirrel Monkey", "Simia sciureus";
        "Chimpanzee", "Pan troglodytes" ]
    |> Map.ofList;;
  \end{verbatim}
\end{frame}

\begin{frame}[fragile]{Maps (2)}
  \begin{itemize}[<+->]
    \item \texttt{.[key]} access
  \end{itemize}
  \begin{verbatim}
  > holidays.["Christmas"];;
  \end{verbatim}
  \pause
  \begin{itemize}
    \item functions in \texttt{Map} module (\texttt{exists}, \texttt{filter},
    \texttt{tryfind}, $\ldots$)
  \end{itemize}
\end{frame}

\begin{frame}{Performance}
  \begin{itemize}[<+->]
    \item immutable red-black trees
    \item search, insert, delete - $O(n)$
  \end{itemize}
\end{frame}

\section{Discriminated Unions}
\frame{\tableofcontents[currentsection]}

\againframe{tuples}

\begin{frame}[fragile]{Unions of Constructors}
  \begin{verbatim}
type switchstate =
    | On
    | Off
    | Adjustable of float
  \end{verbatim}
\end{frame}

\begin{frame}[fragile]{Sum and Product Types}
  \begin{verbatim}
type firstType = 
    | C1
    | C2
type secondType =
    | C21 of bool
    | C22 of bool
    | C23 of bool
type thirdType = { C31 : bool; C31 : bool };;
  \end{verbatim}
  How many values in each type?
\end{frame}

\begin{frame}[fragile]{Union of All Datatypes}
  \begin{itemize}[<+->]
    \item generalize construction
    \item type variables
    \item OCaml vs .NET style
  \end{itemize}
  \pause
  \begin{verbatim}
  type 'a tree =
      | Leaf of 'a
      | Node of 'a tree * 'a tree
  \end{verbatim}
  \begin{verbatim}
  type tree<'a> =
      | Leaf of 'a
      | Node of tree<'a> * tree<'a>
  \end{verbatim}
\end{frame}

\begin{frame}[fragile]{Pattern Matching}
  \begin{verbatim}
  let toggle = function
      | On -> Off
      | Off -> On
      | Adjustable(brightness) -> ...;;
  \end{verbatim}
\end{frame}

\againframe{maybe}

\begin{frame}[fragile]{Known Union Types}
  \begin{verbatim}
  type 'a option =
      | Some of 'a
      | None
  \end{verbatim}
  \pause
  \begin{verbatim}
  type 'a list = 
      | Cons of 'a * 'a list
      | Nil
  \end{verbatim}
\end{frame}

\begin{frame}[fragile]{Propositional Logic (Example)}
  \begin{verbatim}
  type proposition =
      | True
      | Not of proposition
      | And of proposition * proposition
      | Or of proposition * proposition
  \end{verbatim}
  Write \texttt{eval : proposition -> bool}.
\end{frame}

\begin{frame}[fragile]{Propositional Logic (Example)}
  \begin{verbatim}
  let rec eval = function
      | True -> ...
      | Not(prop) -> ...
      | And(prop1, prop2) -> ...
      | Or(prop1, prop2) -> ...
  \end{verbatim}
\end{frame}

\begin{frame}[fragile]{Propositional Logic (Example)}
  \begin{verbatim}
  let rec eval = function
      | True -> true
      | Not(prop) -> ...
      | And(prop1, prop2) -> ...
      | Or(prop1, prop2) -> ...
  \end{verbatim}
\end{frame}

\begin{frame}[fragile]{Propositional Logic (Example)}
  \begin{verbatim}
  let rec eval = function
      | True -> true
      | Not(prop) -> not (eval(prop))
      | And(prop1, prop2) -> ...
      | Or(prop1, prop2) -> ...
  \end{verbatim}
\end{frame}

\begin{frame}[fragile]{Propositional Logic (Example)}
  \begin{verbatim}
  let rec eval = function
      | True -> true
      | Not(prop) -> not (eval(prop))
      | And(prop1, prop2) -> eval(prop1) && eval(prop2)
      | Or(prop1, prop2) -> eval(prop1) || eval(prop2)
  \end{verbatim}
\end{frame}

\begin{frame}[fragile]{Stacks (Example)}
  \begin{itemize}[<+->]
    \item first in - last out
    \item as a list (\texttt{cons}, \texttt{skip 1})
    \item as a proper data structure
  \end{itemize}
  \pause
  \begin{verbatim}
  type 'a stack =
      | EmptyStack
      | StackNode of 'a * 'a stack
  let stack = StackNode(1, StackNode(2, StackNode(3, 
      StackNode(4, StackNode(5, EmptyStack)))))
  \end{verbatim}
\end{frame}

\begin{frame}[fragile]{Stacks (Example)}
  \texttt{head : 'a stack -> 'a}
  \pause
  \begin{verbatim}
  let head = function
          | EmptyStack -> ...
          | StackNode(hd, tl) -> ...
  \end{verbatim}
\end{frame}

\begin{frame}[fragile]{Stacks (Example)}
  \texttt{head : 'a stack -> 'a}
  \begin{verbatim}
  let head = function
          | EmptyStack -> failwith "Empty stack"
          | StackNode(hd, tl) -> ...
  \end{verbatim}
\end{frame}

\begin{frame}[fragile]{Stacks (Example)}
  \texttt{head : 'a stack -> 'a}
  \begin{verbatim}
  let head = function
          | EmptyStack -> failwith "Empty stack"
          | StackNode(hd, tl) -> hd
  \end{verbatim}
\end{frame}

\begin{frame}[fragile]{Stacks (Example)}
  \texttt{tail : 'a stack -> 'a stack}
  \pause
  \begin{verbatim}
  let head = function
          | EmptyStack -> failwith "Empty stack"
          | StackNode(hd, tl) -> tl
  \end{verbatim}
\end{frame}

\begin{frame}[fragile]{Stacks (Example)}
  \texttt{update : int -> 'a -> 'a stack -> 'a stack}
  \pause
  \small
  \begin{verbatim}
  let rec update index value s =
      match index, s with
          | index, EmptyStack -> failwith "Index out of range"
          | 0, StackNode(hd, tl) -> StackNode(value, tl)
          | n, StackNode(hd, tl) -> StackNode(hd, 
              update (index - 1) value tl)
  \end{verbatim}
\end{frame}

\begin{frame}[fragile]{Immutable Data Structures}
  \begin{itemize}[<+->]
    \item no destructive updates
    \item copying and sharing
  \end{itemize}
  \pause
  \begin{verbatim}
         0      1      2      3      4
          ___    ___    ___    ___    ___ 
let x =  |_1_|->|_2_|->|_3_|->|_4_|->|_5_|->Empty
                              ^
                             /
          ___    ___    ___ /
let y =  |_1_|->|_2_|->|_9_|
  \end{verbatim}
\end{frame}

\begin{frame}[fragile]{Stacks (Example)}
  \texttt{append : 'a stack -> 'a stack -> 'a stack}
  \pause
  \small
  \begin{verbatim}
  let rec append x y =
      match x with
          | EmptyStack -> y
          | StackNode(hd, tl) -> StackNode(hd, append tl y)
  \end{verbatim}
\end{frame}

\begin{frame}[fragile]{Stacks (Example)}
  \texttt{map : ('a -> 'b) -> 'a stack -> 'b stack}
  \pause
  \small
  \begin{verbatim}
  let rec map f = function
          | EmptyStack -> EmptyStack
          | StackNode(hd, tl) -> StackNode(f hd, map f tl)
  \end{verbatim}
\end{frame}

\begin{frame}{Queue (Example)}
  \begin{itemize}[<+->]
    \item first in - first out
    \item as a list (\texttt{cons}, \texttt{take (length - 1)})
    \item as a proper data structure containing 2 stacks
  \end{itemize}
  \pause
  Left as exercise
\end{frame}

\begin{frame}[fragile]{Binary Trees (Example)}
  \begin{verbatim}
  type 'a tree =
      | EmptyTree
      | TreeNode of 'a * 'a tree * 'a tree
  \end{verbatim}
  \pause
  Binary Search Tree - left as exercise
\end{frame}

\begin{frame}[fragile]{Binary Trees (Example)}
  \texttt{map : ('a -> 'b) -> 'a tree -> 'b tree}
  \pause
  \small
  \begin{verbatim}
  let rec map f = function
      | EmptyTree -> EmptyTree
      | TreeNode(n, l, r) -> TreeNode(f n, map f l, map f r)
  \end{verbatim}
\end{frame}

\section{Lazy Data Structures}
\frame{\tableofcontents[currentsection]}

\begin{frame}{Efficiency of Eager Immutable Data Structures}
  \begin{itemize}[<+->]
    \item not as efficient as imperative equivalents
    \item appending 2 stacks - $O(n)$ time
    \item laziness - $O(1)$ (amortized) time
  \end{itemize}
\end{frame}

\begin{frame}[fragile]{A Lazy Stack}
  \begin{verbatim}
  type 'a lazyStack =
      | Node of Lazy<'a * 'a lazyStack>
      | EmptyStack
  \end{verbatim}
  \pause
  \begin{verbatim}
  let (|Cons|Nil|) = function
      | Node(item) ->
          let hd, tl = item.Force()
          Cons(hd, tl)
      | EmptyStack -> Nil
  \end{verbatim}
\end{frame}

\begin{frame}[fragile]{A Lazy Stack}
  \begin{verbatim}
  let hd = function
      | Cons(hd, tl) -> hd
      | Nil -> failwith "empty"

  let tl = function
      | Cons(hd, tl) -> tl
      | Nil -> failwith "empty"
\end{verbatim}
\end{frame}

\begin{frame}[fragile]{A Lazy Stack}
  \begin{verbatim}
let rec append x y =
    match x with
    | Cons(hd, tl) -> Node(
        lazy(printfn "appending... got %A" hd;
        hd, append tl y))
    | Nil -> y

let rec iter f = function
    | Cons(hd, tl) -> f(hd); iter f tl
    | Nil -> ()
  \end{verbatim}
\end{frame}

\begin{frame}[fragile]{A Lazy Stack}
  \begin{verbatim}
> let x = cons(1, cons(2, cons(3, cons(4, EmptyStack))));;
val x : int lazyStack = Node <unevaluated>
> let y = cons(5, cons(6, cons(7, EmptyStack)));;
val y : int lazyStack = Node <unevaluated>
> let z = append x y;;
val z : int lazyStack = Node <unevaluated>
\end{verbatim}
\end{frame}

\begin{frame}[fragile]{A Lazy Stack}
  \begin{verbatim}
> hd z;;
appending... got 1
val it : int = 1

> hd (tl (tl z) );;
appending... got 2
appending... got 3
val it : int = 3
\end{verbatim}
\end{frame}

\begin{frame}[fragile]{A Lazy Stack}
  \begin{verbatim}
> iter (fun x -> printfn "%i" x) z;;
1
2
3
appending... got 4
4
5
6
7
  \end{verbatim}
\end{frame}

\begin{frame}{Why Immutability?}
  \begin{itemize}[<+->]
    \item old structure stays
    \item lock-less code
    \item easy concurrency
    \item theorems for free
    \item undo for free
  \end{itemize}
\end{frame}

\begin{frame}{Immutable Data Structures in Imperative Code?}
  Possible
  \pause
  and recommended from time to time.
\end{frame}

\section{Caching Results}
\frame{\tableofcontents[currentsection]}

\begin{frame}{Immutability = Multiple Copies}
  \begin{itemize}[<+->]
    \item modifying a data structure = creating a new copy
    \item garbage-collection
    \item when? how often?
    \item reusing data?
  \end{itemize}
\end{frame}

\begin{frame}{Caching Pointfree Functions}
  \begin{itemize}[<+->]
    \item point-free functions as .NET properties
    \item function is evaluated once
    \item cached result
  \end{itemize}
\end{frame}

\begin{frame}[fragile]{No Caching}
\small
\begin{verbatim}
> let isNebraskaCity_bad city =
    let cities =
        printfn "Creating cities Set"
        ["Bellevue"; "Omaha"; "Lincoln"; "Papillion"]
        |> Set.of_list

    cities.Contains(city);;

val isNebraskaCity_bad : string -> bool

> isNebraskaCity_bad "Lincoln";;
Creating cities Set
val it : bool = true

> isNebraskaCity_bad "Washington";;
Creating cities Set
val it : bool = false
\end{verbatim}
\end{frame}

\begin{frame}[fragile]{Caching}
\small
\begin{verbatim}
> let isNebraskaCity_good =
    let cities =
        printfn "Creating cities Set"
        ["Bellevue"; "Omaha"; "Lincoln"; "Papillion"]
        |> Set.of_list

    fun city -> cities.Contains(city);;
Creating cities Set

val isNebraskaCity_good : (string -> bool)

> isNebraskaCity_good "Lincoln";;
val it : bool = true

> isNebraskaCity_good "Washington";;
val it : bool = false
\end{verbatim}
\end{frame}

\begin{frame}[fragile]{Fibonacci Function}
  \begin{verbatim}
  > let rec fib n =
      if n = 0I then 0I
      elif n = 1I then 1I
      else fib (n - 1I) + fib(n - 2I);;
  \end{verbatim}
  \begin{verbatim}
  > #time;;
  > fib 35I;;
  \end{verbatim}
\end{frame}

\begin{frame}[fragile]{Memoization}
  Keep calculated values in memory.
  \small
  \begin{verbatim}
> let rec fib =
    let dict = new System.Collections.Generic.Dictionary<_,_>()
    fun n ->
        match dict.TryGetValue(n) with
        | true, v -> v
        | false, _ -> 
            let temp =
                if n = 0I then 0I
                elif n = 1I then 1I
                else fib (n - 1I) + fib(n - 2I)
            dict.Add(n, temp)
            temp;;
  \end{verbatim}
\end{frame}

\begin{frame}[fragile]{Generic Memoizer}
\small
  \begin{verbatim}
let memoize f =
    let dict = new System.Collections.Generic.Dictionary<_,_>()
    fun n ->
        match dict.TryGetValue(n) with
        | (true, v) -> v
        | _ ->
            let temp = f(n)
            dict.Add(n, temp)
            temp
 
let rec fib = memoize(fun n ->
    if n = 0I then 0I
    elif n = 1I then 1I
    else fib (n - 1I) + fib (n - 2I) )
  \end{verbatim}
\end{frame}

\begin{frame}{Performance Concerns}
  \begin{itemize}[<+->]
    \item laziness + memoization = big win
    \item tie the knot
    \item computing hashes takes time
    \item always profile
  \end{itemize}
\end{frame}

\begin{frame}[fragile]{Lazy Values and Caching}
  All lazy values are computed only once.
\small
  \begin{verbatim}
> let x = seq {
         for a in 1 .. 5 -> printfn "Got %i" a; a } |> Seq.cache;;
val x : seq<int>

> let y = Seq.take 3 x;;
val y : seq<int>

> Seq.reduce (+) y;;
Got 1
Got 2
Got 3
val it : int = 9

> Seq.reduce (+) y;;
val it : int = 9

> Seq.reduce (+) x;;
Got 4
...
  \end{verbatim}
\end{frame}

\section{Finishing slides}
\frame{\tableofcontents[currentsection]}

\begin{frame}{Recap}
  \begin{itemize}[<+->]
    \item option types
    \item records
    \item sets and maps
    \item discriminated unions
    \item caching
  \end{itemize}
\end{frame}

\begin{frame}{Resources}
  \begin{itemize}
    \item Chris Okasaki - \textit{Functional Data Structures}
    \item Steve Skiena - \textit{The Algorithm Design Manual}
    \item
    http://stackoverflow.com/questions/3208258/memoization-in-haskell/3209189\#3209189
  \end{itemize}
\end{frame}

\begin{frame}[fragile]{Assignment}
  \begin{itemize}
    \item optional
    \item send via mail (\texttt{mihai@rosedu.org})
    \item next talk will start with discussion of solutions
  \end{itemize}
  \pause
  \begin{block}{Joining tables}
  Starting from lists \texttt{(name * age) list}, \texttt{(name * phone) list}
and \texttt{(name * location) list} define \texttt{(name * age * phone *
location) list} containing all values from all 3 lists.
  \end{block}
\end{frame}

\end{document}
