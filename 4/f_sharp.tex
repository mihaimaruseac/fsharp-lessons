\documentclass{beamer}

\usepackage{hyperref}
\mode<presentation>

\title{F\#.4 : Type zoo}
\author{Mihai Maruseac, ROSEdu\\mihai@rosedu.org}

\setbeamertemplate{frametitle continuation}[from second]
\setbeamertemplate{footline}[frame number]

\begin{document}

\maketitle

\begin{frame}
  \tableofcontents
\end{frame}

\begin{frame}{Last time...}
  \begin{itemize}[<+->]
    \item option types
    \item records
    \item sets and maps
    \item discriminated unions
    \item caching
  \end{itemize}
\end{frame}

\begin{frame}{Last Assignment}
  \begin{block}{Joining tables}
  Starting from lists \texttt{(name * age) list}, \texttt{(name * phone) list}
and \texttt{(name * location) list} define \texttt{(name * age * phone *
location) list} containing all values from all 3 lists.
  \end{block}
\end{frame}

\begin{frame}[fragile]{Solution (1)}
  TODO
\end{frame}

\section{Active Patterns}
\frame{\tableofcontents[currentsection]}

\section{Units of Measure}
\frame{\tableofcontents[currentsection]}

\section{Functors}
\frame{\tableofcontents[currentsection]}

\section{Applicative}
\frame{\tableofcontents[currentsection]}

\section{Monads}
\frame{\tableofcontents[currentsection]}

\section{Finishing slides}
\frame{\tableofcontents[currentsection]}

\begin{frame}{Monad Tutorial Fallacy}
\end{frame}

\begin{frame}{Monads in Imperative World}
\end{frame}

\begin{frame}{Typeclassopedia}
\end{frame}

\begin{frame}{Recap}
  \begin{itemize}[<+->]
    \item Active Patterns
    \item Units of Measure
    \item Functors
    \item Applicative
    \item Monads / Computation Expressions
    \item Typeclassopedia
  \end{itemize}
\end{frame}

\begin{frame}{Resources}
  \begin{itemize}
    \item TODO
    \item http://stackoverflow.com/questions/44965/what-is-a-monad
    \item http://www.haskell.org/haskellwiki/Monad\_tutorials\_timeline
  \end{itemize}
\end{frame}

\begin{frame}[fragile]{Assignment}
  \begin{itemize}
    \item optional
    \item send via mail (\texttt{mihai@rosedu.org})
    \item next talk will start with discussion of solutions
  \end{itemize}
  \pause
  \begin{block}{Joining tables}
  TODO
  \end{block}
\end{frame}

\end{document}
