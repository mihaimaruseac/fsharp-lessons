\documentclass{beamer}

\usepackage{hyperref}
\mode<presentation>

\title{F\#.4 : Type zoo}
\author{Mihai Maruseac, ROSEdu\\mihai@rosedu.org}

\setbeamertemplate{frametitle continuation}[from second]
\setbeamertemplate{footline}[frame number]

\begin{document}

\maketitle

\begin{frame}
  \tableofcontents
\end{frame}

\begin{frame}{Last time...}
  \begin{itemize}[<+->]
    \item option types
    \item records
    \item sets and maps
    \item discriminated unions
    \item caching
  \end{itemize}
\end{frame}

\begin{frame}{Last Assignment}
  \begin{block}{Joining tables}
  Starting from lists \texttt{(name * age) list}, \texttt{(name * phone) list}
and \texttt{(name * location) list} define \texttt{(name * age * phone *
location) list} containing all values from all 3 lists.
  \end{block}
\end{frame}

\begin{frame}[fragile]{Solution (1)}
  TODO
\end{frame}

\section{Active Patterns}
\frame{\tableofcontents[currentsection]}

\begin{frame}[fragile]{A Lazy Stack}
  \begin{verbatim}
  type 'a lazyStack =
      | Node of Lazy<'a * 'a lazyStack>
      | EmptyStack
  \end{verbatim}
  \begin{verbatim}
  let (|Cons|Nil|) = function
      | Node(item) ->
          let hd, tl = item.Force()
          Cons(hd, tl)
      | EmptyStack -> Nil
  \end{verbatim}
\end{frame}

\begin{frame}[fragile]{Banana Brackets}
  \begin{itemize}[<+->]
    \item \texttt{(|...|)}
    \item define ad-hoc union data structure
    \item anonymous types
  \end{itemize}
  \pause
  \begin{verbatim}
let (|Even|Odd|) n =
    if n % 2 = 0 then Even else Odd
  \end{verbatim}
  \pause
  \begin{verbatim}
type numKind =
    | Even
    | Odd

let get_choice n =
    if n % 2 = 0 then Even else Odd
  \end{verbatim}
\end{frame}

\begin{frame}[fragile]{Active Patterns with Parameters}
  \begin{verbatim}
let (|SeqNode|SeqEmpty|) s =
    if Seq.isEmpty s then SeqEmpty
    else SeqNode ((Seq.hd s), Seq.skip 1 s)
  \end{verbatim}
  \pause
  \begin{verbatim}
type 'a seqWrapper =
    | SeqEmpty
    | SeqNode of 'a * seq<'a>

let get_choice s =
    if Seq.isEmpty s then SeqEmpty
    else SeqNode ((Seq.hd s), Seq.skip 1 s)
  \end{verbatim}
\end{frame}

\begin{frame}[fragile]{Usage}
  \begin{itemize}
    \item pattern matching
  \end{itemize}
  \begin{verbatim}
> let (|Even|Odd|) n =
      if n % 2 = 0 then Even
      else Odd;;
> let testNum n =
      match n with
      | Even -> printfn "%i is even" n
      | Odd -> printfn "%i is odd" n;;
> testNum 12;;
12 is even
  \end{verbatim}
\end{frame}

\begin{frame}[fragile]{Partial Patterns}
  \tiny
  \begin{verbatim}
> let (|RegexContains|_|) pattern input =
      let matches = System.Text.RegularExpressions.Regex.Matches(input, pattern)
      if matches.Count > 0 then Some [ for m in matches -> m.Value ]
      else None;;

> let testString = function
      | RegexContains "http://\S+" urls -> printfn "Got urls: %A" urls
      | RegexContains "[^@]@[^.]+\.\W+" emails -> printfn "Got email address: %A" emails
      | RegexContains "\d+" numbers -> printfn "Got numbers: %A" numbers
      | _ -> printfn "Didn't find anything.";;

> testString "867-5309, Jenny are you there?";;
Got numbers: ["867"; "5309"]
  \end{verbatim}
\end{frame}

\begin{frame}[fragile]{Partial Patterns :: Translation}
  \tiny
  \begin{verbatim}
type choice =
    | RegexContains of string list

let get_choice pattern input =
    let matches = System.Text.RegularExpressions.Regex.Matches(input, pattern)
    if matches.Count > 0 then Some (RegexContains [ for m in matches -> m.Value ])
    else None

let testString n =
    match get_choice "http://\S+" n with
    | Some(RegexContains(urls)) -> printfn "Got urls: %A" urls
    | None ->
        match get_choice "[^@]@[^.]+\.\W+" n with
        | Some(RegexContains emails) -> printfn "Got email address: %A" emails
        | None ->
            match get_choice "\d+" n with
            | Some(RegexContains numbers) -> printfn "Got numbers: %A" numbers
            | _ -> printfn "Didn't find anything."
  \end{verbatim}
\end{frame}

\begin{frame}[fragile]{Mixing Partial Active Patterns}
  \tiny
  \begin{verbatim}
let (|StartsWith|_|) pat (input : string) = if input.StartsWith(needle) then Some() else None
let (|EndsWith|_|) pat (input : string) = if input.EndsWith(needle) then Some() else None
let (|Equals|_|) x y = if x = y then Some() else None
let (|RegexContains|_|) pat input =
    let matches = System.Text.RegularExpressions.Regex.Matches(input, pat)
    if matches.Count > 0 then Some [ for m in matches -> m.Value ]
    else None

let testString n =
    match n with
    | StartsWith "kitty" () -> printfn "starts with 'kitty'"
    | StartsWith "bunny" () -> printfn "starts with 'bunny'"
    | EndsWith "doggy" () -> printfn "ends with 'doggy'"
    | Equals "monkey" () -> printfn "equals 'monkey'"
    | RegexContains "http://\S+" urls -> printfn "Got urls: %A" urls
    | RegexContains "[^@]@[^.]+\.\W+" emails -> printfn "Got email address: %A" emails
    | RegexContains "\d+" numbers -> printfn "Got numbers: %A" numbers
    | _ -> printfn "Didn't find anything."
  \end{verbatim}
\end{frame}

\begin{frame}{Why}
  \begin{itemize}[<+->]
    \item anonymous types
    \item stop namespace pollution
    \item help type inference
    \item partial patterns $\sim$ \texttt{option} type
    \item mixing partial patterns
    \item easier than calling multiple \texttt{option} functions
    \item can pattern match structures resistant to pattern match (\texttt{Seq})
    \item not restricted to a finite number of cases
    \item mix freely in match statements
  \end{itemize}
\end{frame}

\section{Units of Measure}
\frame{\tableofcontents[currentsection]}

\begin{frame}{Remember Hungarian Notation?}
  \begin{itemize}[<+->]
    \item \texttt{szName}, \texttt{LPINT}, $\ldots$
    \item \textit{kind} + name (\textbf{not} \textit{type})
    \item useful for replacing missing type systems
    \item Systems Hungarian Notation - bad
    \item \textit{Hungarian Notation Is Not Recommended}
  \end{itemize}
\end{frame}

\begin{frame}{Data with Contextual Information}
  \begin{itemize}[<+->]
    \item row and column in Excel
    \item different systems of coordinates: desktop/window/widget
    \item $\ldots$
    \item easy for code reviewers to spot mistakes
    \item why not compiler?
  \end{itemize}
\end{frame}

\begin{frame}[fragile]{Units of Measure}
  \begin{verbatim}
  [<Measure>] type m                  (* meter *)
  [<Measure>] type s                  (* second *)
  [<Measure>] type kg                 (* kilogram *)
  [<Measure>] type N = (kg * m)/(s^2) (* Newtons *)
  [<Measure>] type Pa = N/(m^2)       (* Pascals *)
  \end{verbatim}
\end{frame}

\begin{frame}[fragile]{Units of Measure}
  \begin{verbatim}
  > let distance = 100.0<m>;;
  > let time = 5.0<s>;;
  > let speed = distance / time;;
  \end{verbatim}
\end{frame}

\begin{frame}[fragile]{Units of Measure}
  \begin{verbatim}
  [<Measure>] type C
  [<Measure>] type F

  let to_fahrenheit (x : float<C>) =
      x * (9.0<F>/5.0<C>) + 32.0<F>

  let to_celsius (x : float<F>) =
      (x - 32.0<F>) * (5.0<C>/9.0<F>)
  \end{verbatim}
\end{frame}

\begin{frame}[fragile]{Units of Measure}
  \small
  \begin{verbatim}
  > speed 20.0<m> 4.0<m>;; (* boom! *)

    speed 20.0<m> 4.0<m>;;
    --------------^^^^^^

  stdin(39,15): error FS0001: Type mismatch. Expecting a
          float<s>
  but given a
          float<m>.
  \end{verbatim}
\end{frame}

\begin{frame}[fragile]{Units of Measure for Integrals}
  \begin{verbatim}
  [<Measure>] type col
  [<Measure>] type row
  let colOffset (a : int<col>) (b : int<col>) = a - b
  let rowOffset (a : int<row>) (b : int<row>) = a - b
  \end{verbatim}
\end{frame}

\begin{frame}[fragile]{Dimensionless Units}
  \begin{itemize}[<+->]
    \item use \texttt{<1>} for type
  \end{itemize}
  \begin{verbatim}
  let to_meters x = x * 1<m>
  let of_meters (x : float<m>) = float x
  let of_meters2 (x : float<m>) = x / 1.0<m>
  \end{verbatim}
\end{frame}

\begin{frame}[fragile]{Transparent Operations}
  \small
  \begin{verbatim}
let vanillaFloats = [10.0; 15.5; 17.0]
let lengths = [ for a in [2.0; 7.0; 14.0; 5.0] -> a * 1.0<m> ]
let masses = [ for a in [55.54; 79.01; 35.90] -> a * 1.0<kg> ]

let average (l : float<'u> list) =
    let sumtuple (sum, count) x = sum + x, count + 1.0<_>
    let sum, count = l |> List.fold sumtuple (0.0<_>, 0.0<_>)
    sum / count;;

> average vanillaFloats, average lengths, average masses;;
val it : float * float<m> * float<kg> =
  (14.16666667, 7.0, 56.8166667)
  \end{verbatim}
\end{frame}

\begin{frame}[fragile]{In Data Types}
  \begin{itemize}
    \item not to be confunded with generics/type variables
  \end{itemize}
  \begin{verbatim}
> type triple<[<Measure>] 'a> =
    { a : float<'a>; b : float<'a>; c : float<'a>};;

> { a = 7.0<kg>; b = -10.5<_>; c = 0.5<_> };;
val it : triple<kg> = {a = 7.0;
                       b = -10.5;
                       c = 0.5;}
  \end{verbatim}
\end{frame}

\begin{frame}{Behind the Scenes}
  \begin{itemize}[<+->]
    \item looking like proper types
    \item but there are no operation on types (\texttt{*,/,\^})
    \item used only at compile time (type inference)
    \item erased afterwards
    \item not kept in assemblies
    \item cannot determine measure type at runtime
    \item cannot be used directly as a type parameter
  \end{itemize}
\end{frame}

\section{Functors}
\frame{\tableofcontents[currentsection]}

\begin{frame}[fragile]{A Zoo of \texttt{map}s}
  \small
  \begin{itemize}[<+->]
    \item \texttt{List.map}, \texttt{Seq.map}
    \item \texttt{Option.map}
    \item \texttt{Set.map}
    \item \texttt{Stack.map}, \texttt{Tree.map}
  \end{itemize}
  \pause
  \begin{verbatim}
  let rec map f = function
          | EmptyStack -> EmptyStack
          | StackNode(hd, tl) -> StackNode(f hd, map f tl)
  \end{verbatim}
  \begin{verbatim}
  let rec map f = function
      | EmptyTree -> EmptyTree
      | TreeNode(n, l, r) -> TreeNode(f n, map f l, map f r)
  \end{verbatim}
\end{frame}

\begin{frame}{Functors}
  \begin{itemize}[<+->]
    \item first intuition:
    \item "containers" of some sort
    \item ability to apply a function \textbf{uniformly} to \textbf{all}
    elements in container
    \item second intuition:
    \item computational context
    \item function composition
    \item in the end:
    \item data structures supporting \texttt{map}-like operations
  \end{itemize}
\end{frame}

\begin{frame}{Laws}
  \begin{itemize}[<+->]
    \item \texttt{map id == \pause id}
    \pause
    \item \texttt{map (f . g) == \pause map f . \pause map g}
    \pause
    \item each type has \textbf{only} one \textit{valid} instance of \texttt{map}
    \item usually, second law comes for free
    \item Category Theory
  \end{itemize}
\end{frame}

\begin{frame}[fragile]{Example (1)}
  \small
  \begin{verbatim}
  let rec map f = function
      | EmptyTree -> EmptyTree
      | TreeNode(n, l, r) -> TreeNode(f n, map f l, r)
  \end{verbatim}
  \pause
  \textbf{Not correct} (not on all data contained)
\end{frame}

\begin{frame}[fragile]{Example (2)}
  \tiny
  \begin{verbatim}
  let rec map f = function
      | EmptyTree -> EmptyTree
      | TreeNode(n, l, r) -> TreeNode (f n, TreeNode(f n, map f l, map f r), EmptyTree)
  \end{verbatim}
  \pause
  \textbf{Not correct} (first law)
\end{frame}

\begin{frame}{Why}
  \begin{itemize}[<+->]
    \item proofs of correctness
    \item focus on \textbf{what}, not \textit{how}
    \item free instances
    \item basis for complex type structures
  \end{itemize}
\end{frame}

\section{Applicative}
\frame{\tableofcontents[currentsection]}

\begin{frame}{Another View on \texttt{map}}
    \begin{itemize}[<+->]
      \item \texttt{map : ('a -> 'b) -> 'a option -> 'b option}
      \item remember curry vs. uncurry functions
      \item \texttt{map : ('a -> 'b) -> ('a option -> 'b option)}
      \item \textit{lift} a function to \texttt{option} context
    \end{itemize}
\end{frame}

\begin{frame}{Applicative Functors}
  \begin{itemize}[<+->]
    \item \texttt{('a -> 'b) option -> 'a option -> 'b option}
    \item functions in context
    \item keep same context
    \item Applicative Programming Style
    \item not used in F\#
  \end{itemize}
\end{frame}

\section{Monads}
\frame{\tableofcontents[currentsection]}

\section{Finishing slides}
\frame{\tableofcontents[currentsection]}

\begin{frame}{Monad Tutorial Fallacy}
\end{frame}

\begin{frame}{Monads in Imperative World}
\end{frame}

\begin{frame}{Typeclassopedia}
\end{frame}

\begin{frame}{Recap}
  \begin{itemize}[<+->]
    \item Active Patterns
    \item Units of Measure
    \item Functors
    \item Applicative
    \item Monads / Computation Expressions
    \item Typeclassopedia
  \end{itemize}
\end{frame}

\begin{frame}{Resources}
  \begin{itemize}
    \item http://www.joelonsoftware.com/articles/Wrong.html
    \item http://stackoverflow.com/questions/44965/what-is-a-monad
    \item http://www.haskell.org/haskellwiki/Monad\_tutorials\_timeline
    \item http://www.haskell.org/haskellwiki/Typeclassopedia
  \end{itemize}
\end{frame}

\begin{frame}[fragile]{Assignment}
  \begin{itemize}
    \item optional
    \item send via mail (\texttt{mihai@rosedu.org})
    \item next talk will start with discussion of solutions
  \end{itemize}
  \pause
  \begin{block}{Joining tables}
  TODO
  \end{block}
\end{frame}

\end{document}
