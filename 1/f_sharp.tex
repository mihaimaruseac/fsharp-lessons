\documentclass{beamer}

\usepackage{hyperref}
\mode<presentation>

\title{F\#.1 : Introduction. Basic Concepts}
\author{Mihai Maruseac, ROSEdu\\mihai@rosedu.org}

\setbeamertemplate{frametitle continuation}[from second]
\setbeamertemplate{footline}[frame number]

\pgfdeclareimage[height=7cm]{langrankbig}{img/langrankbig}
\pgfdeclareimage[height=7cm]{langs}{img/langs}

\begin{document}

\maketitle

\begin{frame}{About me}
  \begin{itemize}
    \item MSc student, AI, UPB (+T.A.)
    \item AI, Functional Programming, Compilers, Operating Systems
    \item Haskell, Python, C
    \item Pardus, Utopia
    \item IRC, FB, Twitter, etc
  \end{itemize}
\end{frame}

\begin{frame}{Language popularity}
  \center
  \pgfuseimage{langrankbig}
\end{frame}

\begin{frame}{Language matrix}
  \center
  \pgfuseimage{langs}
\end{frame}

\begin{frame}{Why?}
  \begin{block}{Alan Perlis}
   \textit{A language that doesn't affect the way you think about programming, is not
   worth knowing.}
  \end{block}
  \pause
  \begin{block}{Ludwig Wittgenstein}
   \textit{The limits of my language mean the limits of my world.}
  \end{block}
  \pause
  \begin{block}{Benjamin Lee Whorf}
   \textit{Language shapes the way we think, and determines what we can think
   about.}
  \end{block}
\end{frame}

\begin{frame}{History}
  \begin{itemize}[<+->]
    \item 1930s, Alonzo Church, $\lambda$-calculus
    \item 1950s, John McCarthy, \texttt{LISP}
    \begin{itemize}
      \item \texttt{LISP} $\rightarrow$ \texttt{Scheme}, \texttt{Dylan},
      \texttt{Scala}, \texttt{Racket}
    \end{itemize}
    \item 1960s, \texttt{IPL}, \texttt{APL}
    \item 1970s, \texttt{FP}, John Bachus
    \begin{itemize}
      \item \texttt{Can Programming Be Liberated From the von Neumann Style? A
      Functional Style and its Algebra of Programs}
      \item \texttt{APL}, \texttt{FP} $\rightarrow$ \texttt{J}
    \end{itemize}
    \item 1970s, Robin Milner, \texttt{ML}
    \item 1970s, David Turner, \texttt{Miranda}
    \item 1990s, committee, \texttt{Haskell}
    \begin{itemize}
      \item open standard for functional programming research
    \end{itemize}
  \end{itemize}
\end{frame}

\begin{frame}{ML Languages}
  \begin{itemize}
    \item \texttt{OCaml}, \texttt{StandardML}
    \item \texttt{Miranda}
    \item \texttt{Haskell}
    \pause
    \item \texttt{F\#} (\texttt{.NET}, \texttt{C\#}, 2005, Microsoft Research)
  \end{itemize}
\end{frame}

\begin{frame}{Features}
  \begin{itemize}
    \item \texttt{.NET}
    \item 3 paradgims encompased: functional, imperative, object-oriented
    \item statically typed
    \item declarative style
  \end{itemize}
\end{frame}

\begin{frame}[fragile]{Functional Programming}
  \begin{itemize}
    \item every input has a corresponding input
    \item $ f(x) = x + 1$
    \pause
    \item only if pure!!
    \begin{itemize}
      \item no side effects
      \item order does not matter
      \item easy concurrency
    \end{itemize}
    \item functions are values
    \item function composition
    \item $g(x) = x * 3$, $g(f(x)) = 3 * x + 3$
  \end{itemize}
  \begin{verbatim}
  > let f x = x + 1;;
  > let g x = x * 3;;
  > g(f(100));;
  val it : int = 303
  \end{verbatim}
\end{frame}

\begin{frame}[fragile]{Static Typing}
  \begin{itemize}
    \item everything has a type
    \item powerful type inferrence
    \item proof lies in code (iif using proper type encodings)
    \begin{itemize}
      \item code compiles $\rightarrow$ 95\% bug-free
    \end{itemize}
  \end{itemize}
  \begin{verbatim}
  > f;;
  val f : int -> int
  > g;;
  val g : int -> int
  > g(f(100));;
  val it : int = 303
  \end{verbatim}
\end{frame}

\begin{frame}[fragile]{Types as documentation}
  Guess function from type

  \begin{verbatim}
  val f : 'a -> 'a
  val g : bool -> bool
  val h : 'a -> 'b -> 'a
  val i : int -> int -> int
  val j : float -> float -> float
  \end{verbatim}

  \pause

  \begin{verbatim}
  > let f x = x;;
  > let g x = not x;;
  > let h x y = x;;
  > let i x y = x + y;;
  > let j x y:float = x + y;;
  \end{verbatim}
\end{frame}

\begin{frame}[fragile]{Declarative Programming}
  \begin{itemize}
    \item focus on \textit{what} not \textit{how}
    \item code reuse (function composition, anonymous functions, closures,
    etc.)
  \end{itemize}
  \begin{verbatim}
> let rec qs l =
    match l with
    | x::t -> let l,u =
      List.partition ((>=)x) t in (qs l) @ x :: (qs u) 
    | [] -> [];;

val qs : 'a list -> 'a list when 'a : comparison

> qs [4; 3; 2; 5];;
val it : int list = [2; 3; 4; 5]
  \end{verbatim}
\end{frame}

\begin{frame}[fragile]{Sorting in different languages :: C\#}
\tiny{
  \begin{verbatim}
using System;
using System.Collections.Generic;
namespace QuickSort
{
    class Program
    {
        private static List<int> quicksort(List<int> arr)
        {
            List<int> loe = new List<int>(), gt = new List<int>();
            if (arr.Count < 2) return arr;
            int pivot = arr.Count / 2;
            int pivot_val = arr[pivot];
            arr.RemoveAt(pivot);
            foreach (int i in arr)
            {
                if (i <= pivot_val)
                    loe.Add(i);
                else if (i > pivot_val)
                    gt.Add(i);
            }
            List<int> resultSet = new List<int>();
            resultSet.AddRange(quicksort(loe));
            gt.Add(pivot_val);
            resultSet.AddRange(quicksort(gt));
            return resultSet;
        }
    }
}
  \end{verbatim}
}
\end{frame}

\begin{frame}[fragile]{Sorting in different languages :: Scheme, Haskell F\#}
\tiny{
  \begin{verbatim}
(define (quicksort l gt?)
  (if (null? l) '()
      (append (quicksort (filter (lambda (x) (gt? (car l) x)) (cdr l)) gt?)
              (list (car l))
              (quicksort (filter (lambda (x) (not (gt? (car l) x))) (cdr l)) gt?))))
  \end{verbatim}

  \begin{verbatim}
qsort []     = []
qsort (x:xs) = qsort [y | y <- xs, y < x] ++ [x] ++ qsort [y | y <- xs, y >= x]
  \end{verbatim}

  \begin{verbatim}
let rec qsort = function
    [] -> []
    | hd :: tl ->
        let less, greater = List.partition ((>=) hd) tl
        List.concat [qsort less; [hd]; qsort greater]
  \end{verbatim}
}
\end{frame}

\begin{frame}[fragile]{F\# Variables}
  \texttt{let varName [: type] = value}
  \tiny{
  \begin{verbatim}
  > let answer = 42;;
  val answer : int = 42

  > let pi : double = 3.14159;;
  val pi : double = 3.14159

  > let f = fun x -> x;;
  val f : 'a -> 'a

  > let g x = x;;
  val g : 'a -> 'a

  > let list = [1; 2; 3; 4];;
  val list : int list = [1; 2; 3; 4]

  > let tuple = 1,2, 3, "ana";;
  val tuple : int * int * int * string = (1, 2, 3, "ana")
  \end{verbatim}
  }
\end{frame}

\begin{frame}[fragile]{F\# Functions}
  2 types:
  \begin{itemize}
    \item explicit arguments
    \begin{verbatim}
  > let g x = x;;
  val g : 'a -> 'a
    \end{verbatim}
    \item lambda-abstractions (anonymous)
    \begin{verbatim}
    > fun x -> x;;
    val it : 'a -> 'a = <fun:clo@116-3>
    \end{verbatim}
  \end{itemize}
\end{frame}

\begin{frame}[fragile]{Function sequencing}
  \begin{itemize}
    \item pipe-forward \texttt{|>}
    \item function-compose \texttt{>>}
  \end{itemize}
  \begin{verbatim}
  > let f x = 2 * x;;
  val f : int -> int

  > let g = f >> f;;
  val g : (int -> int)

  > g 5;;
  val it : int = 20
  > f 5 |> f;;
  val it : int = 20
  > f 5 |> fun x -> x;;
  val it : int = 10
  \end{verbatim}
\end{frame}

\begin{frame}[fragile]{Pattern matching}
  Makes reading code easier (no more \texttt{if}s and \texttt{elif}s)
  \begin{verbatim}
  > let f x = if x <= 0 then 0 else x - 1;;

  > let g x = match x with
              0 -> 0
              | x -> x - 1;;

  > let h x = match x with
              | 0 -> 0
              | x -> x - 1;;
  \end{verbatim}
\end{frame}

\begin{frame}[fragile]{Recursive definitions}
  Names are visible \textbf{after} their definition ends.
  \begin{verbatim}
  > let rec fact x = match x with
                     0 -> 1
                     | x -> x * fact (x - 1);;
  val fact : int -> int

  > fact 3;;
  val it : int = 6
  \end{verbatim}
\end{frame}

\begin{frame}[fragile]{Lists[1]}
  \begin{itemize}
    \item bread and butter of functional programming
    \item ordered collection of values
    \item \texttt{cons} (\texttt{::}), \texttt{head}, \texttt{tail}
  \end{itemize}
  \begin{verbatim}
  > [];;
  val it : 'a list = []
  > 2 :: [];;
  val it : int list = [2]
  > 2 :: (3 :: []);;
  > val it : int list = [2; 3]
  > 2 :: (3 :: [42; 45]);;
  val it : int list = [2; 3; 42; 45]
  \end{verbatim}
\end{frame}

\begin{frame}[fragile]{Lists[2]}
  \begin{itemize}
    \item Faster generation: ranges and init
    \item List from string
  \end{itemize}
  \begin{verbatim}
  > List.init 5 (fun x -> x * x);;
  > val it : int list = [0; 1; 4; 9; 16]

  > [0 .. 5];;
  val it : int list = [0; 1; 2; 3; 4; 5]

  > List.ofSeq "Winter is coming";;
  val it : char list =
    ['W'; 'i'; 'n'; 't'; 'e'; 'r'; ' '; 'i'; 's'; ' ';
       'c'; 'o'; 'm'; 'i'; 'n'; 'g']
  \end{verbatim}
\end{frame}

\begin{frame}[fragile]{Lists[3]}
  \begin{itemize}
    \item Same type restriction
  \end{itemize}
  \tiny{
  \begin{verbatim}
> [3; 3.14];;

  [3, 3.14];;
  ----^^^^
stdin(195,5): error FS0001: This expression was expected to have type
    int    
but here has type
    float    

> [3; 4; "random"];;

  [3; 4; "random"];;
  -------^^^^^^^^
stdin(196,8): error FS0001: This expression was expected to have type
    int    
but here has type
    string
  \end{verbatim}
}
\end{frame}

\begin{frame}[fragile]{Lists[4]}
  \begin{itemize}
    \item Element access
  \end{itemize}
\tiny{
  \begin{verbatim}
> List.nth [1;2;3;4] 3;;
val it : int = 4
  \end{verbatim}
  \pause
  \begin{verbatim}
> List.nth [1;2;3;4] 5;;
System.ArgumentException: The index was outside the range of elements in the list.
Parameter name: index
   at Microsoft.FSharp.Collections.ListModule.Get[T](FSharpList`1 list, Int32 index)
   at <StartupCode$FSI_0110>.$FSI_0110.main@()
Stopped due to error
  \end{verbatim}
}
\end{frame}

\begin{frame}[fragile]{Lists[5]}
  \begin{itemize}
    \item Pattern matching on constructors
  \end{itemize}
  \tiny{
  \begin{verbatim}
> let isEmpty l = match l with
                  [] -> true
                  | x::xs -> false;; //or | _ -> false
val isEmpty : 'a list -> bool

> isEmpty [2;3;4];;
val it : bool = false
> isEmpty [2];;
val it : bool = false
> isEmpty [];;
val it : bool = true
  \end{verbatim}
  }
\end{frame}

\begin{frame}[fragile]{Lists[6]}
  \begin{itemize}
    \item Using recursion on lists
  \end{itemize}
\tiny{
  \begin{verbatim}
> let rec len l = match l with
                  [] -> 0
                  | _ :: xs -> 1 + len xs;;

val len : 'a list -> int

> len [1;2;3;4];;
val it : int = 4
> len [];;
val it : int = 0
  \end{verbatim}
  \pause
  \begin{verbatim}
> let rec sum l = match l with
                  [] -> 0
                  | x :: xs -> x + sum xs;;

val sum : int list -> int

> sum [1 .. 5];;
val it : int = 15
  \end{verbatim}
}
\end{frame}

\begin{frame}[fragile]{Tuples}
  \begin{itemize}
    \item also known as product types
  \end{itemize}
\tiny{
  \begin{verbatim}
> 1,2,3;;
val it : int * int * int = (1, 2, 3)

> (1, 2, 3);;
val it : int * int * int = (1, 2, 3)

> 1, 3.14, "ana are mere";;
val it : int * float * string = (1, 3.14, "ana are mere")

> let f x = match x with (x,_,_) -> x;;
val f : 'a * 'b * 'c -> 'a

> f 3,4,5;;

  let f x = match x with (x,_,_) -> x;;
  --^

stdin(265,3): error FS0001: This expression was expected to have type
    'a * 'b * 'c    
but here has type
    int

> f (3, 4, 5);;
> val it : int = 3
  \end{verbatim}
}
\end{frame}

\begin{frame}[fragile]{Assignment}
  \begin{itemize}
    \item optional
    \item send via mail (\texttt{mihai@rosedu.org})
    \item next talk will start with discussion of solutions
  \end{itemize}
  \pause
  \begin{itemize}
    \item Implement a \textit{Caesar cipher}
    \item For input \texttt{cipher "hello" 13}
    \item Should produce \texttt{uryyb}
  \end{itemize}
\end{frame}

\end{document}
